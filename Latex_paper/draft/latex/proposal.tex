\documentclass[a4paper,12pt]{report}

\usepackage[utf8]{inputenc}
% \usepackage[Vietnamese]{babel}
\usepackage{titling}
%package the indent the first line in latex
\usepackage{indentfirst}
\usepackage{graphicx}
\usepackage{ragged2e}
\usepackage{ragged2e}
\usepackage{afterpage}

%adding bibliography
% \usepackage[backend=biber]{biblatex}
% \addbibresource{test.bib}


\graphicspath{{/home/pj/Documents/proposal/}}

%making the numbering in Roman format 
\renewcommand\thesection{\Roman{section}}
\renewcommand\thesubsection{\roman{subsection}}

\setlength{\droptitle}{-8cm}

\usepackage{tikz}

\pretitle{
    \begin{tikzpicture}[remember picture,overlay]
    \node[anchor=north west,yshift=-1.5pt,xshift=1pt]%
        at (current page.north west)
        {\includegraphics{vgu_logo}};
    \end{tikzpicture}
}
\posttitle{}

\begin{titlepage}
	\title{ MUSIC SHEET UNDERSTANDING AND TONE MODULATION}
	\author{}
\end{titlepage}


\begin{document}

\afterpage{\null\newpage}

\maketitle

\tableofcontents

\clearpage

\section{Research team members}
\begin{itemize}
	\item Team Leader:      \hfill Truong Minh Khoa
	\item Programmer: 		\hfill Dinh Cong Minh
	\item ML Engineer:		\hfill Nguyen Tho Anh Khoa
	\item Writer/Editor:	\hfill Huynh Minh Triet
\end{itemize}


\section{Disclaimer} 
This report is a product of our team's work, unless otherwise referenced. In
addition, all opinions, results, conclusions, and recommendations are of our own
and may not represent the policies or opinions of the Vietnamese-German
University's Department of Engineering or the University as a whole. 

\clearpage

\section{Abstract}

\section{Introduction}

The topic of recognizing musical sheets, i.e., Optical Music Recognition (OMR),
is not a novel field of research. The term OMR first appeared in a paper written
by MIT scientists in the 60s.  During the last three decades until now, OMR is
an ever increasingly developing field and is capable of solving many problems
that involves with music

More specifically, the current OMR systems of today are capable enough to
recognize a printed musical sheet and digitize it. The resulting output could be
a .midi file, or other types of sound files such as .way, .mp3. The vast
majority of those researches are dedicated for the common user, even for users
who are not educated on musical theory, but there is still a lack of product
that can be used for professional or enthusiast musicians. In reality, a common
problem that is encountered is the modulation of music tones, i.e., up or down
semitone, tone for the whole music sheet. Currently in order to obtain a music
sheet with a few tones higher or lower the musicsian has to manually retype the
entire musical sheet by hand, which is labor intensive.

\subsection{Our solution}
We propose an algorithm that can take it a pdf file as it input, then shift the
tone of the entire song up or down to the number of tones, semitones according
to the musician's need. 


\clearpage

\section{Proposed Method}
To acheive our desire result we need to first remove the lines on each staff,
second we translate the pictorial images of the note into text based notation,
finally we will run our algorithm to shift our tones 

\subsection{Line Removal}




% \printbibliography


\end{document}
